\documentclass{article}
\usepackage{enumitem}
\usepackage{hyperref}

\begin{document}

\title{Документация для проекта BrickGame v2.0 aka Snake}
\date{}
\maketitle

\section{Введение}

Проект BrickGame v2.0 aka Snake представляет собой реализацию классической игры Змейка с использованием языка программирования C++ и библиотеки Qt для десктопного интерфейса.

\section{Структура проекта}

\subsection{Библиотека Snake (\texttt{src/brick\_game/snake})}

\begin{itemize}[label=--]
    \item Файлы с исходным кодом, реализующие логику игры Змейка.
    \item Основные функции для взаимодействия с игровым полем, управлением змейкой и обработкой ввода пользователя.
    \item Реализация конечного автомата для формализации логики игры.
\end{itemize}

\subsection{Десктопный интерфейс (\texttt{src/gui/desktop})}

\begin{itemize}[label=--]
    \item Файлы с исходным кодом, отвечающие за визуализацию игры с использованием библиотеки Qt.
    \item Реализация отрисовки игрового поля, управления вводом пользователя и отображения текущего состояния игры.
\end{itemize}

\section{Сборка проекта}

Проект использует систему сборки \texttt{make} с Makefile, включающим следующие цели:

\begin{itemize}
    \item \texttt{all}: Сборка проекта.
    \item \texttt{install}: Установка программы в систему.
    \item \texttt{uninstall}: Удаление программы из системы.
    \item \texttt{clean}: Очистка временных файлов и папок.
    \item \texttt{clear}: Полная очистка файлов и папок.
    \item \texttt{dvi}: Создание файла DVI.
    \item \texttt{dist}: Создание архива, содержащего необходимые файлы для сборки и использования программы.
\end{itemize}

\section{Требования к среде выполнения}

Проект предполагает использование языка программирования C++ стандарта C++17, компилятора g++ и библиотеки Qt для десктопного интерфейса, а также языка программирования C, компилятора gcc и библиотеки ncurses для варианта программы с использованием терминального интерфейса.

\section{Инструкции по установке и запуску}

\begin{enumerate}
    \item \textbf{Установка зависимостей:}
        \begin{itemize}
            \item Убедитесь, что у вас установлен компилятор g++ (или gcc для терминального интерфейса).
            \item Установите библиотеку Qt (или ncurses для терминального интерфейса).
        \end{itemize}
    \item \textbf{Сборка проекта:}
        \begin{itemize}
            \item Выполните \texttt{make} для сборки проекта.
        \end{itemize}
    \item \textbf{Установка:}
        \begin{itemize}
            \item Выполните \texttt{make install} для установки программы в систему.
        \end{itemize}
    \item \textbf{Запуск:}
        \begin{itemize}
            \item Выполните \texttt{./build/snake_dsk} для запуска программы.
        \end{itemize}
\end{enumerate}

\section{Использование программы}

\begin{enumerate}
    \item \textbf{Управление:}
        \begin{itemize}
            \item Используйте стрелки влево и вправо для поворота змейки.
            \item Нажмите клавишу пробел для ускорения перемешения.
            \item Для снятия и постановки игры на паузу используйте ENTER.
            \item Для выхода из игры используйте ESC.
        \end{itemize}
    \item \textbf{Механики игры:}
        \begin{itemize}
            \item Поворот змейки и перемещение по игровому полю.
            \item Ускорение движения.
            \item Случайная генерация и поедание яблок.
            \item Рост змейки.
            \item Подсчет очков и хранения масимального счета.
            \item Увеличение уровней и скорости.
        \end{itemize}
    \item \textbf{Завершение игры:}
        \begin{itemize}
            \item Игра завершается при столкновении с границей игрового поля или телом змейки, либо при достижении максимального размера змейки.
        \end{itemize}
\end{enumerate}

\section{Тестирование}

Проект включает в себя unit-тесты с использованием библиотеки GTest. Покрытие библиотеки тестами составляет не менее 80\%.

\end{document}
