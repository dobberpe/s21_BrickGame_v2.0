\documentclass{article}
\usepackage{enumitem}
\usepackage{hyperref}

\begin{document}

\title{Документация для проекта BrickGame v1.0 aka Tetris}
\date{}
\maketitle

\section{Введение}

Проект BrickGame v1.0 aka Tetris представляет собой реализацию классической аркадной игры Тетрис с использованием языка программирования C и библиотеки ncurses для терминального интерфейса.

\section{Структура проекта}

\subsection{Библиотека Tetris (\texttt{src/brick\_game/tetris})}

\begin{itemize}[label=--]
    \item Файлы с исходным кодом, реализующие логику игры Тетрис.
    \item Основные функции для взаимодействия с игровым полем, управлением фигурами и обработкой ввода пользователя.
    \item Реализация конечного автомата для формализации логики игры.
\end{itemize}

\subsection{Терминальный интерфейс (\texttt{src/gui/cli})}

\begin{itemize}[label=--]
    \item Файлы с исходным кодом, отвечающие за визуализацию игры в терминале с использованием библиотеки ncurses.
    \item Реализация отрисовки игрового поля, управления вводом пользователя и отображения текущего состояния игры.
\end{itemize}

\section{Сборка проекта}

Проект использует систему сборки \texttt{make} с Makefile, включающим следующие цели:

\begin{itemize}
    \item \texttt{all}: Сборка проекта.
    \item \texttt{install}: Установка программы в систему.
    \item \texttt{uninstall}: Удаление программы из системы.
    \item \texttt{clean}: Очистка временных файлов и папок.
    \item \texttt{dvi}: Создание файла DVI.
    \item \texttt{dist}: Создание архива, содержащего необходимые файлы для сборки и использования программы.
\end{itemize}

\section{Требования к среде выполнения}

Проект предполагает использование языка программирования C11, компилятора gcc и библиотеки ncurses для терминального интерфейса.

\section{Инструкции по установке и запуску}

\begin{enumerate}
    \item \textbf{Установка зависимостей:}
        \begin{itemize}
            \item Убедитесь, что у вас установлен компилятор gcc.
            \item Установите библиотеку ncurses (если необходимо).
        \end{itemize}
    \item \textbf{Сборка проекта:}
        \begin{itemize}
            \item Выполните \texttt{make all} для сборки проекта.
        \end{itemize}
    \item \textbf{Установка:}
        \begin{itemize}
            \item Выполните \texttt{make install} для установки программы в систему.
        \end{itemize}
    \item \textbf{Запуск:}
        \begin{itemize}
            \item Выполните \texttt{make run} для запуска программы.
        \end{itemize}
\end{enumerate}

\section{Использование программы}

\begin{enumerate}
    \item \textbf{Управление:}
        \begin{itemize}
            \item Используйте стрелки влево и вправо для перемешения фигуры в по горизонтали.
            \item Нажмите клавишу вниз для ускорения падения фигуры.
            \item Другие клавиши для дополнительных действий.
            \item Для поворота фигуры используйте ENTER.
            \item Для снятия и постановки игры на паузу используйте пробел.
            \item Для выхода из игры используйте ESC.
        \end{itemize}
    \item \textbf{Механики игры:}
        \begin{itemize}
            \item Вращение и перемещение фигур.
            \item Ускорение падения фигуры.
            \item Показ следующей фигуры.
            \item Уничтожение заполненных линий.
            \item Завершение игры при достижении верхней границы игрового поля.
        \end{itemize}
    \item \textbf{Завершение игры:}
        \begin{itemize}
            \item Игра завершается, когда достигнута верхняя граница игрового поля.
        \end{itemize}
\end{enumerate}

\section{Тестирование}

Проект включает в себя unit-тесты с использованием библиотеки check. Покрытие библиотеки тестами составляет не менее 80\%.

\end{document}
